%% For double-blind review submission
%\documentclass[sigplan,10pt,review,anonymous]{acmart}\settopmatter{printfolios=true}
%% For single-blind review submission
\documentclass[sigplan,10pt,review]{acmart}\settopmatter{printfolios=true}
%% For final camera-ready submission
%\documentclass[sigplan,10pt]{acmart}\settopmatter{}

%% Note: Authors migrating a paper from traditional SIGPLAN
%% proceedings format to PACMPL format should change 'sigplan,10pt' to
%% 'acmlarge'.

\usepackage[T1]{fontenc}
\usepackage[utf8]{inputenc}
\usepackage{xcolor}
\usepackage{forest}
\usepackage{amssymb}
\usepackage{mathtools}
\usepackage{marvosym}
\newcommand{\MVUparrow}{\,\scalebox{1}[1.4]{\rotatebox[origin=c]{90}{\MVRightarrow}}}
\newcommand{\MVDnarrow}{\,\scalebox{1}[1.4]{\rotatebox[origin=c]{-90}{\MVRightarrow}}}
\usepackage{semantic}
\usepackage[normalem]{ulem}
\usepackage{enumitem}
\setlist{leftmargin=5.5mm}
\usepackage{lipsum}

\newcommand{\sub}[2]{\mathclose{\{^{#2\!\!\!}\reflectbox{$\!\smallsetminus$}_{\!#1}\hspace{-.2mm}\}}}
\usepackage{mylhs2tex}


%% Some recommended packages.
\usepackage{booktabs}   %% For formal tables:
                        %% http://ctan.org/pkg/booktabs
\usepackage{subcaption} %% For complex figures with subfigures/subcaptions
                        %% http://ctan.org/pkg/subcaption
%%%

\newcommand{\xone}[1]{\mathop{\raisebox{-.3ex}{$\xrightarrow{\raisebox{-.5ex}{\scriptsize{\ensuremath{#1}}}}$}}}
\newcommand{\xoneb}[1]{\mathop{\raisebox{-.3ex}{$\xrightharpoonup{\raisebox{-.5ex}{\scriptsize{\ensuremath{#1}}}}_{\!\!\!\textsc{b}}$}}}
\newcommand{\OM}[0]{\ensuremath{\mathcal{O\!M}}}

\makeatletter\if@ACM@journal\makeatother
%% Journal information (used by PACMPL format)
%% Supplied to authors by publisher for camera-ready submission
\acmJournal{PACMPL}
\acmVolume{1}
\acmNumber{1}
\acmArticle{1}
\acmYear{2017}
\acmMonth{1}
\acmDOI{10.1145/nnnnnnn.nnnnnnn}
\startPage{1}
\else\makeatother
%% Conference information (used by SIGPLAN proceedings format)
%% Supplied to authors by publisher for camera-ready submission
\acmConference[Haskell'17]{ACM SIGPLAN Haskell Symposium}{September 07--08, 2017}{Oxford, UK}
\acmYear{2017}
\acmISBN{978-x-xxxx-xxxx-x/YY/MM}
\acmDOI{10.1145/nnnnnnn.nnnnnnn}
\startPage{1}
\fi


%% Copyright information
%% Supplied to authors (based on authors' rights management selection;
%% see authors.acm.org) by publisher for camera-ready submission
\setcopyright{none}             %% For review submission
%\setcopyright{acmcopyright}
%\setcopyright{acmlicensed}
%\setcopyright{rightsretained}
%\copyrightyear{2017}           %% If different from \acmYear


%% Bibliography style
\bibliographystyle{ACM-Reference-Format}
%% Citation style
%% Note: author/year citations are required for papers published as an
%% issue of PACMPL.
%\citestyle{acmauthoryear}  %% For author/year citations
%\citestyle{acmnumeric}     %% For numeric citations
%\setcitestyle{nosort}      %% With 'acmnumeric', to disable automatic
                            %% sorting of references within a single citation;
                            %% e.g., \cite{Smith99,Carpenter05,Baker12}
                            %% rendered as [14,5,2] rather than [2,5,14].
%\setcitesyle{nocompress}   %% With 'acmnumeric', to disable automatic
                            %% compression of sequential references within a
                            %% single citation;
                            %% e.g., \cite{Baker12,Baker14,Baker16}
                            %% rendered as [2,3,4] rather than [2-4].



\begin{document}

%% Title information
%% \title[Generating Certificates for non-Open Bisimilar Processes]{
%%  	Generating Certificates for Pi-Calculus Processes
%%  	Distingishable by Open Bisimulation}         %% [Short Title] is optional;
%% \title[Generating Distinguishing Formulae for the pi-Calculus]{
%% 	Generating Distinguishing Formulae for Open Bisimilarity for the pi-Calculus}         %% [Short Title] is optional;

\title{Generating Witness of Non-Bisimilarity for the pi-Calculus}

%% when present, will be used in
                                        %% header instead of Full Title.
\titlenote{Draft submitted to Haskell'17}  %% \titlenote is optional;
                                        %% can be repeated if necessary;
                                        %% contents suppressed with 'anonymous'
% \subtitle{Subtitle}                     %% \subtitle is optional
% \subtitlenote{with subtitle note}       %% \subtitlenote is optional;
                                        %% can be repeated if necessary;
                                        %% contents suppressed with 'anonymous'


%% Author information
%% Contents and number of authors suppressed with 'anonymous'.
%% Each author should be introduced by \author, followed by
%% \authornote (optional), \orcid (optional), \affiliation, and
%% \email.
%% An author may have multiple affiliations and/or emails; repeat the
%% appropriate command.
%% Many elements are not rendered, but should be provided for metadata
%% extraction tools.

\author{Ki Yung Ahn}
% \authornote{with author1 note}          %% \authornote is optional;
                                          %% can be repeated if necessary
\orcid{nnnn-nnnn-nnnn-nnnn}             %% \orcid is optional
\affiliation{
%  \position{Position1}
%  \department{Department1}              %% \department is recommended
  \institution{Nanyang Technological University}            %% \institution is required
%  \streetaddress{Street1 Address1}
%  \city{City1}
%  \state{State1}
%  \postcode{PostCode1}
  \country{Singapore}
}
\email{yaki@ntu.edu.sg}          %% \email is recommended

\author{Ross Horne}
% \authornote{with author2 note}          %% \authornote is optional;
                                          %% can be repeated if necessary
\orcid{nnnn-nnnn-nnnn-nnnn}             %% \orcid is optional
\affiliation{
%  \position{Position2a}
%  \department{Department2a}             %% \department is recommended
  \institution{Nanyang Technological University}           %% \institution is required
%  \streetaddress{Street Address}
%  \city{City}
%  \state{State}
%  \postcode{PostCode}
  \country{Singapore}
}
\email{rhorne@ntu.edu.sg}         %% \email is recommended

\author{Alwen Tiu}
% \authornote{with author2 note} %% \authornote is optional; can be repeated if necessary
%\orcid{nnnn-nnnn-nnnn-nnnn}      %% \orcid is optional
\affiliation{
%  \position{Position2a}
%  \department{Department2a}             %% \department is recommended
  \institution{Nanyang Technological University}           %% \institution is required
%  \streetaddress{Street2a Address2a}
%  \city{City}
%  \state{State}
%  \postcode{PostCode}
  \country{Singapore}
}
\email{atiu@ntu.edu.sg}         %% \email is recommended



%% Paper note
%% The \thanks command may be used to create a "paper note" ---
%% similar to a title note or an author note, but not explicitly
%% associated with a particular element.  It will appear immediately
%% above the permission/copyright statement.
%\thanks{with paper note}                %% \thanks is optional
                                        %% can be repeated if necesary
                                        %% contents suppressed with 'anonymous'


%% Abstract
%% Note: \begin{abstract}...\end{abstract} environment must come
%% before \maketitle command
\begin{abstract}In the logic programming paradigm, it is difficult to develop
an elegant solution for generating distinguishing formulae that
witness the failure of open-bisimilarity between two pi-calculus processes;
this was unexpected because the semantics of the pi-calculus and
open bisimulation have already been elegantly specified
in higher-order logic programming systems.
Our solution using Haskell defines the formulae generation as a tree transformation from
the forest of all nondeterministic bisimulation steps to a pair of distinguishing formulae.
Thanks to laziness in Haskell, only the necessary paths demanded by the tree transformation
function are generated. Our work demonstrates that Haskell and its libraries
provide an attractive platform for symbolically analyzing equivalence properties of
labeled transition systems in an environment sensitive setting.

\end{abstract}


%% 2012 ACM Computing Classification System (CSS) concepts
%% Generate at 'http://dl.acm.org/ccs/ccs.cfm'.
\begin{CCSXML}
<ccs2012>
<concept>
<concept_id>10003752.10003753.10003761.10003764</concept_id>
<concept_desc>Theory of computation~Process calculi</concept_desc>
<concept_significance>500</concept_significance>
</concept>
<concept>
<concept_id>10003752.10003790.10003793</concept_id>
<concept_desc>Theory of computation~Modal and temporal logics</concept_desc>
<concept_significance>300</concept_significance>
</concept>
<concept>
<concept_id>10003752.10003790.10003795</concept_id>
<concept_desc>Theory of computation~Constraint and logic programming</concept_desc>
<concept_significance>300</concept_significance>
</concept>
<concept>
<concept_id>10003752.10010124.10010131.10010134</concept_id>
<concept_desc>Theory of computation~Operational semantics</concept_desc>
<concept_significance>300</concept_significance>
</concept>
<concept>
<concept_id>10003752.10010124.10010138.10010142</concept_id>
<concept_desc>Theory of computation~Program verification</concept_desc>
<concept_significance>100</concept_significance>
</concept>
<concept>
<concept_id>10011007.10011006.10011008.10011009.10011012</concept_id>
<concept_desc>Software and its engineering~Functional languages</concept_desc>
<concept_significance>500</concept_significance>
</concept>
<concept>
<concept_id>10003033.10003039.10003041.10003042</concept_id>
<concept_desc>Networks~Protocol testing and verification</concept_desc>
<concept_significance>100</concept_significance>
</concept>
<concept>
<concept_id>10003033.10003039.10003041.10003043</concept_id>
<concept_desc>Networks~Formal specifications</concept_desc>
<concept_significance>100</concept_significance>
</concept>
</ccs2012>
\end{CCSXML}

\ccsdesc[500]{Theory of computation~Process calculi}
\ccsdesc[300]{Theory of computation~Modal and temporal logics}
\ccsdesc[300]{Theory of computation~Constraint and logic programming}
\ccsdesc[300]{Theory of computation~Operational semantics}
\ccsdesc[100]{Theory of computation~Program verification}
\ccsdesc[500]{Software and its engineering~Functional languages}
\ccsdesc[100]{Networks~Protocol testing and verification}
\ccsdesc[100]{Networks~Formal specifications}
%% End of generated code


%% Keywords
%% comma separated list
%% \keywords is optional
\keywords{process calculus, observational equivalence,
labeled transition systems, open bisimulation, modal logic, dynamic logic,
distinguishing formula, Haskell, lazy evaluation, name binding,
constraint programming, nondeterministic programming}

%% \maketitle
%% Note: \maketitle command must come after title commands, author
%% commands, abstract environment, Computing Classification System
%% environment and commands, and keywords command.
\maketitle

%%%%% Section on Introduction
\input{intro}

%%%%% Section on Syntax
\input{PiCalc} %%%%% \label{sec:syntax}  PiCalc.tex <=== PiCalc.lhs


\section{Labeled Transition Semantics}
\label{sec:lts}
We discuss implementations of the labeled transition rules in Figure~\ref{fig:lts}.
There are two versions:
the first implements the transition step in a fixed world (Section~\ref{sec:lts:ids})
and the second implements the transition step considering all possible worlds
(Section~\ref{sec:lts:open}).
%%%% The former can be viewed as an optimized version of the latter
%%%% restricted to the fixed world given by the identity substitution.
%%%% Both are used in the implementation of open bisimulation in Section~\ref{sec:bisim}.
%
\input{IdSubLTS} %%%%% \label{sec:lts:ids}  IdSubLTS.tex <=== IdSubLTS.lhs
\input{OpenLTS} %%%%% \label{sec:lts:open}  OpenLTS.tex <=== OpenLTS.lhs


%%%%%% OpenBisim contains three sections for Bisim, DFgen and Discussion
\input{OpenBisim}
%%%% \label{sec:bisim}   %% Open Bisimulation
%%%% \label{sec:df}      %% Distinguishing Formulae Generation
%%%% \label{sec:discuss} %% Discussions


%%%%% Section on Related Work
\section{Related Work}
\label{sec:relwork}
In this section, we discuss 
nondeterministic programming using monads (Section~\ref{sec:relwork:monad}),
bisimulation and its characterizing logic (Section~\ref{sec:relwork:logic}),
and related tools (Section~\ref{sec:relwork:tools}).
\vspace*{-1ex}
\subsection{Monadic encodings of Nondeterminism}
\label{sec:relwork:monad}
\citet{Wadler85listm} modeled nondeterminism with a list monad.
Monadic encodings of more sophisticated features involving nondeterminism
(e.g., \cite{FisOleSha09,Hinze00bmt,KisShaFriSab05logict}) have been developed
and applied to various domains (e.g., \cite{ChaGuoKohLoc98,Schrijvers09mcp}) afterwards.
\citet{FisOleSha09} developed a custom monadic datatype for lazy nondeterministic programming.
Their motivation was to find a way combine three desirable features found in
functional logic programming~\cite{Hanus10lea,LopHer99toy,TolSerNit04} and
probabilistic programming~\cite{ErwKol06pfp,Kiselyov16hakaru10}
-- lazyness, sharing (memoization), and nondeterminism, which are known to be tricky
to combine in functional programming. Having two versions of transitions
(Figures~\ref{fig:IdSubLTS} and \ref{fig:OpenLTS}) in our implementation
was to avoid an instance of undesirable side effects from this trickiness --
naive combination of laziness and nondeterminism causing needless traversals.
We expect our code duplication can be lifted by adopting
such a custom nondeterministic monad.% in place of the list monad.
\vspace*{-1ex}
\subsection{Bisimulation and its Characterizing Logic}
\label{sec:relwork:logic}
%%% Bound variables in the $\pi$-calculus introduce subtlety on how to instantiate
%%% those bound variables, hence, lead to the development of different notions of
%%% labeled semantics: \emph{early} and \emph{late} semantics \cite{Milner92picalcII}
%%% were developed prior to the \emph{open} semantics. Bisimulations based on early and late
%%% semantics are not closed under substitutions, hence, not closed under
%%% input-prefixes.
%%% % That is, they cannot capture process equivalence across
%%% % different environments,
%%% This falls short from the motivation of the $\pi$-calculus
%%% to model the \emph{mobility} of processes where channel names can be provided
%%% as inputs during the process execution. This shortcoming has been addressed by
%%% open bisimulation~\cite{Sangiorgi96acta}, which is closed under substitutions,
%%% hence, capturing the notion of equivalence across all environments
%%% (or all possible worlds).
%%%
Hennessy--Milner Logic (HML)~\cite{HenMil80hml} is a classical
characterizing logic for the Calculus of Communicating Systems (CCS)~\cite{Mil82ccs}.
The duality between diamond and box modalities related by negation 
(i.e., $[a]f \equiv \neg\langle a\rangle(\neg f)$ and $\langle a\rangle f \equiv \neg[a](\neg f)$)
holds in HML. This duality continues to hold in the characterizing logics for
early and late bisimulation for the $\pi$-calculus~\cite{MilParWal93lm}.
Presence of this duality makes it easy to obtain the distinguishing formula
for the opposite side by negation.
There have been attempts \cite{TiuMil10,ParBorEriGutWeb15} on developing
a characterizing logic for open bisimulation, but it has not been correctly
established until our recent development of \OM~\cite{AhnHorTiu17corr}.
Our logic \OM\ captures the intuitionistic nature of the open semantics, which
has a natural possible worlds interpretation typically found in Kripke-style model
of intuitionistic logic. The classical duality between diamond and box modalities
no longer hold in \OM. This is why we needed to keep track of pairs of formulae
for both sides %% and have extra sideway traversals
during our distinguishing formulae generation in Section~\ref{sec:df}.%
\vspace*{-1ex}
\subsection{Tools for Checking Process Equivalence}
\label{sec:relwork:tools}
There are various existing tools that implement bisimulation or
other equivalence checking for variants and extensions of the $\pi$-calculus.
None of these tools generate distinguishing formulae for open bisimulation. 
The Mobility Workbench~\cite{VicMol94mwb} is a tool for the $\pi$-calculus
with features including open bisimulation checking.
It is developed using an old version of SML/NJ.
SPEC~\cite{TiuNamHor16spec} is security protocol verifier based on
open bisimulation checking \cite{TiuDaw10} for the spi-calculus~\cite{Abadi97ccs}.
%% , which is a variant of $\pi$-calculus whose terms are enriched by security primitives.
The core of SPEC including open bisimulation checking is specified by
higher-order logic predicates in Bedwyr \cite{Bedwyr07} and the user interface
is implemented in OCaml.
ProVerif~\cite{BlaFou05} is another security protocol verifier based on
the applied $\pi$-calculus~\cite{AbaFou01appi}. It implements a sound
approximation of observational equivalence, but not bisimulation. 

%% which is a generic framework of
%% $\pi$-calculus insatiable by supplying the term syntax and its
%% rewriting rules. ProVerif is implemented in OCaml.
%% The notion of open bisimilarity is not trivially applicable to 
%% the applied $\pi$-calculus due to its feature of active substitution,
%% although there have been some developments~\cite{ZhuGuWu08}.

There are few tools using Haskell for process equivalence.
Most relevant work to our knowledge is the symbolic (early) bisimulation for
LOTOS \cite{CalSha01lotos}, which is a message passing process algebra
similar to value-passing variant of CCS but with distinct features
including multi-way synchronization. Although not for equivalence checking,
%(nor for any other verification purposes),
\citet{Renzy14phi} implemented an interpreter that can be used as
a playground for executing applied $\pi$-calculus processes to
communicate with actual HTTP servers and clients over the internet.

 %%%% \label{sec:relwork}


\section{Conclusion}
\label{sec:concl}
We implemented automatic generation of modal logic formulae that witness
non-open bisimilarity of processes in the $\pi$-calculus.
These formulae can serve as certificates of process inequivalence,
which can be validated with an existing satisfaction checker for the modal logic \OM.
Our implementation enjoys the benefits of laziness, nondeterministic monad, and
first-class constraints; which are well known benefits of constraint programming
in Haskell. Laziness and monadic abstraction allows us to view all possible
control flow of nondeterminism as lazy generated trees, so that we can define
formula generation as a tree transformation. First-class constraints allows us
to manage information of possible worlds. Our problem setting particularly well
highlights these benefits because we needed additional information outside
the control flow of a usual bisimulation check. Our application of Haskell to
distinguishing formula generation demonstrates that Haskell and its ecosystem
are equipped with attractive features for analyzing equivalence properties of
labeled transition systems in an environment sensitive (or knowledge aware) setting.

%% Acknowledgments
\begin{acks}                            %% acks environment is optional
                                        %% contents suppressed with 'anonymous'
  %% Commands \grantsponsor{<sponsorID>}{<name>}{<url>} and
  %% \grantnum[<url>]{<sponsorID>}{<number>} should be used to
  %% acknowledge financial support and will be used by metadata
  %% extraction tools.
  This material is based upon work supported by the
  \grantsponsor{MoEsg}{Ministry of Education, Singapore}{https://www.moe.gov.sg/}
  under Grant No.~\grantnum{MoEsg}{MOE2014-T2-2-076}.
  % Grant No.~\grantnum{MoEsg}{nnnnnnn} and
  %% Any opinions, findings, and
  %% conclusions or recommendations expressed in this material are those
  %% of the author and do not necessarily reflect the views of the
  %% Ministry of Education, Singapore.
\end{acks}


%% Bibliography
\bibliography{biblio}


%%%%%%%%% %% Appendix
%%%%%%%%% \appendix
%%%%%%%%% \section{Appendix}
%%%%%%%%% 
%%%%%%%%% Text of appendix \ldots

\end{document}
