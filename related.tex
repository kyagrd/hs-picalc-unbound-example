\section{Related Work}
\label{sec:relwork}
In this section, we discuss 
nondeterministic programming using monads (Section~\ref{sec:relwork:monad}),
bisimulation and its characterizing logic (Section~\ref{sec:relwork:logic}),
and related tools (Section~\ref{sec:relwork:tools}).
\vspace*{-1ex}
\subsection{Monadic encodings of Nondeterminism}
\label{sec:relwork:monad}
\citet{Wadler85listm} modeled nondeterminism with a list monad.
Monadic encodings of more sophisticated features involving nondeterminism
(e.g., \cite{FisOleSha09,Hinze00bmt,KisShaFriSab05logict}) have been developed
and applied to various domains (e.g., \cite{ChaGuoKohLoc98,Schrijvers09mcp}) afterwards.
\citet{FisOleSha09} developed a custom monadic datatype for lazy nondeterministic programming.
Their motivation was to find a way combine three desirable features found in
functional logic programming~\cite{Hanus10lea,LopHer99toy,TolSerNit04} and
probabilistic programming~\cite{ErwKol06pfp,Kiselyov16hakaru10}
-- lazyness, sharing (memoization), and nondeterminism, which are known to be tricky
to combine in functional programming. Having two versions of transitions
(Figures~\ref{fig:IdSubLTS} and \ref{fig:OpenLTS}) in our implementation
was to avoid an instance of undesirable side effects from this trickiness --
naive combination of laziness and nondeterminism causing needless traversals.
We expect our code duplication can be lifted by adopting
such a custom nondeterministic monad.% in place of the list monad.
\vspace*{-1ex}
\subsection{Bisimulation and its Characterizing Logic}
\label{sec:relwork:logic}
%%% Bound variables in the $\pi$-calculus introduce subtlety on how to instantiate
%%% those bound variables, hence, lead to the development of different notions of
%%% labeled semantics: \emph{early} and \emph{late} semantics \cite{Milner92picalcII}
%%% were developed prior to the \emph{open} semantics. Bisimulations based on early and late
%%% semantics are not closed under substitutions, hence, not closed under
%%% input-prefixes.
%%% % That is, they cannot capture process equivalence across
%%% % different environments,
%%% This falls short from the motivation of the $\pi$-calculus
%%% to model the \emph{mobility} of processes where channel names can be provided
%%% as inputs during the process execution. This shortcoming has been addressed by
%%% open bisimulation~\cite{Sangiorgi96acta}, which is closed under substitutions,
%%% hence, capturing the notion of equivalence across all environments
%%% (or all possible worlds).
%%%
Hennessy--Milner Logic (HML)~\cite{HenMil80hml} is a classical
characterizing logic for the Calculus of Communicating Systems (CCS)~\cite{Mil82ccs}.
The duality between diamond and box modalities related by negation 
(i.e., $[a]f \equiv \neg\langle a\rangle(\neg f)$ and $\langle a\rangle f \equiv \neg[a](\neg f)$)
holds in HML. This duality continues to hold in the characterizing logics for
early and late bisimulation for the $\pi$-calculus~\cite{MilParWal93lm}.
Presence of this duality makes it easy to obtain the distinguishing formula
for the opposite side by negation.
There have been attempts \cite{TiuMil10,ParBorEriGutWeb15} on developing
a characterizing logic for open bisimulation, but it has not been correctly
established until our recent development of \OM~\cite{AhnHorTiu17corr}.
Our logic \OM\ captures the intuitionistic nature of the open semantics, which
has a natural possible worlds interpretation typically found in Kripke-style model
of intuitionistic logic. The classical duality between diamond and box modalities
no longer hold in \OM. This is why we needed to keep track of pairs of formulae
for both sides %% and have extra sideway traversals
during our distinguishing formulae generation in Section~\ref{sec:df}.%
\vspace*{-1ex}
\subsection{Tools for Checking Process Equivalence}
\label{sec:relwork:tools}
There are various existing tools that implement bisimulation or
other equivalence checking for variants and extensions of the $\pi$-calculus.
None of these tools generate distinguishing formulae for open bisimulation. 
The Mobility Workbench~\cite{VicMol94mwb} is a tool for the $\pi$-calculus
with features including open bisimulation checking.
It is developed using an old version of SML/NJ.
SPEC~\cite{TiuNamHor16spec} is security protocol verifier based on
open bisimulation checking \cite{TiuDaw10} for the spi-calculus~\cite{Abadi97ccs}.
%% , which is a variant of $\pi$-calculus whose terms are enriched by security primitives.
The core of SPEC including open bisimulation checking is specified by
higher-order logic predicates in Bedwyr \cite{Bedwyr07} and the user interface
is implemented in OCaml.
ProVerif~\cite{BlaFou05} is another security protocol verifier based on
the applied $\pi$-calculus~\cite{AbaFou01appi}. It implements a sound
approximation of observational equivalence, but not bisimulation. 

%% which is a generic framework of
%% $\pi$-calculus insatiable by supplying the term syntax and its
%% rewriting rules. ProVerif is implemented in OCaml.
%% The notion of open bisimilarity is not trivially applicable to 
%% the applied $\pi$-calculus due to its feature of active substitution,
%% although there have been some developments~\cite{ZhuGuWu08}.

There are few tools using Haskell for process equivalence.
Most relevant work to our knowledge is the symbolic (early) bisimulation for
LOTOS \cite{CalSha01lotos}, which is a message passing process algebra
similar to value-passing variant of CCS but with distinct features
including multi-way synchronization. Although not for equivalence checking,
%(nor for any other verification purposes),
\citet{Renzy14phi} implemented an interpreter that can be used as
a playground for executing applied $\pi$-calculus processes to
communicate with actual HTTP servers and clients over the internet.

